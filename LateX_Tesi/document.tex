Making sense of astrocytic calcium signals - from acquisition to interpretation 


\begin{thebibliography}{50}
	\bibitem{1}
	\textit{A. Semyanov, C. Henneberger, A. Agarwal, Making sense of astrocytic calcium signals - from acquisition to interpretation (2020)}
	
	\bibitem{2}
	\textit{A. Etkin, T. Egner, R. Kalisch, Emotional processing in anterior
		cingulate and medial prefrontal cortex (2011)}
	
	\bibitem{3}
	\textit{M. Carlén, What constitutes the prefrontal cortex?  (2017)}
	
	\bibitem{4}
	\textit{C. Zheng, Y. Huang, B. Bo, L. Wei, Z. Liang, Z. Wang, Projection from the Anterior Cingulate Cortex to the Lateral Part of Mediodorsal Thalamus Modulates Vicarious Freezing Behavior  (2020)}
	
	\bibitem{5}
	\textit{M. Carrillo, Y. Han,
		F. Migliorati, M. Liu,
		V. Gazzola, C. Keysers, Article
		Emotional Mirror Neurons in the Rat’s Anterior
		Cingulate Cortex (2019)}
	
	\bibitem{6}
	\textit{S.Banks, K. Eddy, M. Angstadt, P. Nathan, K. Phan, Amygdala–frontal connectivity during emotion regulation (2007)}
	
	\bibitem{7}
	\textit{D. Harari, R. Abramovich, A. Zozulya, Bridging the species divide: transgenic mice humanized for type-I interferon response  (2014)}
	
	\bibitem{8}
	\textit{D. Scheggia, F. Managò, F. Maltese, F. Papaleo, Somatostatin interneurons in the prefrontal cortex
		control affective state discrimination in mice (2019)}
	
	\bibitem{9}
	\textit{L. Mariotti, G. Losi, A. Lia, Interneuron-specific signaling evokes distinctive
		somatostatin-mediated responses in adult cortical
		astrocytes (2018)}
	
	\bibitem{10}
	\textit{V. Ferretti, F. Maltese,
		G. Contarini, Oxytocin Signaling in the Central Amygdala
		Modulates Emotion Discrimination in Mice (2019)}
	
	\bibitem{11}
	\textit{https://www.inscopix.com/}
	
	\bibitem{12}
	\textit{T. Patriarchi, J. Cho, K. Merten, Ultrafast neuronal imaging of dopamine dynamics with designed genetically encoded sensors (2018)}
	
	\bibitem{13}
	\textit{
		N. Frost, A. Haggart, V. Sohal,
		Dynamic patterns of correlated activity in the prefrontal1
		cortex encode information about social behavior (2020)}
	
	\bibitem{14}
	\textit{S. Wass, M. Whitehorn,  Interpersonal neural entrainment during
		early social interaction (2014)}
	
	\bibitem{15}
	\textit{Y. Pan, G. Novembre, B. Song, Dual brain stimulation enhances interpersonal
		learning through spontaneous movement synchrony (2020)}
	
	\bibitem{16}
	\textit{L. Kingsbury, S. Huang, J. Wang,
		K. Gu, P. Golshani, Correlated Neural Activity and Encoding of Behavior
		across Brains of Socially Interacting Animals (2019)}
	
	\bibitem{17}
	\textit{J. Wichern, Applied Multivariate Statistical Analysis}
	
	\bibitem{18}
	\textit{C. Cutts, S. Eglen, Detecting pairwise correlations in spike trains: an objective comparison of methods and application to the study of retinal waves (2014)}
	
	\bibitem{19}
	\textit{L. Barnett, A. Seth, The MVGC multivariate Granger causality toolbox: A new approach to Granger-causal inference (2014)}
	
	\bibitem{20}
	\textit{W. Chen, B. Anderson, Solutions of Yule‐Walker Equations for Singular AR Processes (2011)}
	
	\bibitem{21}
	\textit{J. Hamilton, Time series analysis. Princeton, NJ: Princeton University Press (1994)}
	
	\bibitem{22}
	\textit{A. Edwards, Likelihood (expanded edition). Baltimore: Johns Hopkins UniversityPress (1992)}
	
	\bibitem{23}
	\textit{S. Wilks, The large-sample distribution of the likelihood ratio for testing compositehypotheses (1938)}
	
	\bibitem{24}
	\textit{S. Ayash, U. Schmitt, M. Müller, Chronic social defeat-induced social avoidance as a proxy of stress resilience in mice involves conditioned learning  (2019)}
	
	\bibitem{25}
	\textit{C. Batson, M. Polycarpou, E. Harmon-Jones, Empathy and attitudes: can feeling for a member of a stigmatized group improve feelings toward the group?  (1997)}
	
	\bibitem{26}
	\textit{D. Brucks,
		A.von Bayern, Parrots Voluntarily Help Each Other to Obtain Food
		Rewards (2020)}
	
	\bibitem{27}
	\textit{C. Krupenye, J. Tan, B. Hare, Bonobos voluntarily hand food to others
		but not toys or tools (2018)}
	
	\bibitem{28}
	\textit{
		I. Bartal, J. Decety, P. Mason, Empathy and pro-social behavior in rats  (2011)}
	
	\bibitem{29}
	\textit{Dal Monte, Chu, Fagan, Chang, Specialized medial prefrontal-amygdala coordination in other-regarding decision preference (2020)}
	
	\bibitem{30}
	\textit{A. Felix-Ortiz, A. Burgos-Robles, N. Bhagat, Bidirectional modulation of anxiety-related and social behaviors by amygdala projections to the medial prefrontal cortex  (2016)}
	
	\bibitem{31}
	\textit{F. de Waal, S. Preston, Mammalian empathy: behavioural manifestations and neural basis (2017)}
	
	\bibitem{32}
	\textit{H. Vries, J. Stevens, H. Vervaecke, Measuring and testing the steepness of dominance hierarchies (2006)}
	
	\bibitem{33}
	\textit{Q. Zhou, A. Nemes, D. Lee, Chemogenetic silencing of hippocampal neurons suppresses epileptic neural circuits (2019)}
	
	\bibitem{34}
	\textit{L. Allsop, R. Wichmann, Corticoamygdala Transfer of Socially Derived Information Gates Observational Learning (2018)}
	
	\bibitem{35}
	\textit{R. sacco, G. Guidoboni, A. Mauri, A Comprehensive Physically Based Approach to Modeling in Bioengineering and Life Sciences (2019)}
	
	\bibitem{36}
	\textit{M. islam, Einstein–Smoluchowski Diffusion Equation: A Discussion (2006)}
	
	\bibitem{37}
	\textit{A. Hodgkin, A. Huxley, A quantitative description of membrane current and its application to conduction 
		and excitation in nerve (1952)}
	
	\bibitem{38}
	\textit{G. Ermentrout, D. Terman, Mathematical Foundations of Neuroscience (2010)}
	
	\bibitem{39}
	\textit{A. Quarteroni, R. Sacco, Matematica Numerica (2014)}
	
	\bibitem{40}
	\textit{A. Quarteroni, Numerical Models for Differential Problems (2017)}
	
	\bibitem{41}
	\textit{N. Vogt, Optogenetic inhibition (2020)}
	
		\bibitem{42}
	\textit{https://studymat.in/structure-of-a-neuron/}
	
	\bibitem{43}
	\textit{https://www.researchgate.net/figure/A-schematic-of-an-action-potential-When-a-stimulus-is-applied-an-action-potential-is_fig1_335650473}
	
	\bibitem{44}
	\textit{https://studymat.in/structure-of-a-neuron/}

\end{thebibliography}

